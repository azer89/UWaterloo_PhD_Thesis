
%%%%%%%%%%%%%%%%%%%%%%%%%%%%%%%%%%%%%%%%%%%%%%%%%%%%%%%%%%
\chapter{Related Work}
\label{chapter_related_work}
%%%%%%%%%%%%%%%%%%%%%%%%%%%%%%%%%%%%%%%%%%%%%%%%%%%%%%%%%%

\mynote{
\begin{packeditems}
\item More than a literature review
\item Organize related work - impose structure
\item Be clear as to how previous work being described relates to your own. This is not just a list of the related work, you must describe how it is related. How is similar? How is it different?
\item The reader should not be left wondering why you've described something!!
\item Critique the existing work - Where is it strong where is it weak? What are the unreasonable/undesirable assumptions?
\item Identify opportunities for more research (i.e., your thesis) Are there unaddressed, or more important related topics?
\item After reading this chapter, one should understand the motivation for and importance of your thesis
\item You should clearly and precisely define all of the key concepts dealt with in the rest of the thesis, and teach the reader what s/he needs to know to understand the rest of the thesis.
\end{packeditems}
}

%%%%%%%%%%%%%%%%%%%%%%%%%%%%%%%%%%%%%%%%%%%%%%%%%%%%%%%%%%
\textbf{Rigid Packings}
%%%%%%%%%%%%%%%%%%%%%%%%%%%%%%%%%%%%%%%%%%%%%%%%%%%%%%%%%%

%%%%%%%%%%%%%%%%%%%%%%%%%%%%%%%%%%%%%%%%%%%%%%%%%%%%%%%%%%
\textbf{Non-rigid Packings}
%%%%%%%%%%%%%%%%%%%%%%%%%%%%%%%%%%%%%%%%%%%%%%%%%%%%%%%%%%

%%%%%%%%%%%%%%%%%%%%%%%%%%%%%%%%%%%%%%%%%%%%%%%%%%%%%%%%%%
\textbf{Tilings}
%%%%%%%%%%%%%%%%%%%%%%%%%%%%%%%%%%%%%%%%%%%%%%%%%%%%%%%%%%

%%%%%%%%%%%%%%%%%%%%%%%%%%%%%%%%%%%%%%%%%%%%%%%%%%%%%%%%%%
\textbf{3D Packings}
%%%%%%%%%%%%%%%%%%%%%%%%%%%%%%%%%%%%%%%%%%%%%%%%%%%%%%%%%%

%%%%%%%%%%%%%%%%%%%%%%%%%%%%%%%%%%%%%%%%%%%%%%%%%%%%%%%%%%
\textbf{Texture Synthesis}
%%%%%%%%%%%%%%%%%%%%%%%%%%%%%%%%%%%%%%%%%%%%%%%%%%%%%%%%%%

%%%%%%%%%%%%%%%%%%%%%%%%%%%%%%%%%%%%%%%%%%%%%%%%%%%%%%%%%%
\textbf{Animated Packings}
%%%%%%%%%%%%%%%%%%%%%%%%%%%%%%%%%%%%%%%%%%%%%%%%%%%%%%%%%%

%%%%%%%%%%%%%%%%%%%%%%%%%%%%%%%%%%%%%%%%%%%%%%%%%%%%%%%%%%
\textbf{Packings on Manifolds}
%%%%%%%%%%%%%%%%%%%%%%%%%%%%%%%%%%%%%%%%%%%%%%%%%%%%%%%%%%


\textbf{Rigid packing:} The ever-popular Lloyd's method~\cite{McCool1992}
is an iterative
process for creating a perceptually even distribution of points, and has been
used in various forms in procedural packing methods.
Hausner~\cite{Hausner2001} 
used a variant of Lloyd's method to pack oriented rectangles into a container 
region, simulating the appearance of traditional mosaics.  
Hiller et al.~\cite{Hiller2003} extended Lloyd's method to distribute polygonal
elements instead of points, reducing the overlaps in Hausner's approach.
Dalal et al.~\cite{Dalal2006} used an FFT-based image correlation to reposition
elements iteratively, which could be seen as making more effective use of negative
space, and permitting non-convex elements to interlock more than they did in
earlier methods.

Some past work has sought to adapt example-based texture synthesis methods
from raster images to vector graphics, producing distributions of rigidly transformed elements
that mimic the statistics of an exemplar.  Barla et al.~\cite{Barla2006} and
Ijiri et al.~\cite{Ijiri2008} use a growth model that copies small neighbourhoods
from the exemplar into a larger output texture.  AlMeraj et al.~\cite{AlMeraj2013}
stamp out copies of the exemplar and discard overlapping elements.
Hurtut et al.~\cite{Hurtut2009} develop a statistical sampling method based
on multitype point processes.  
These techniques are all concerned with replicating
the uneven element distribution in the exemplar, without regard for negative space.

\textbf{Dense packing and tessellations:} 
Gal et al.~\cite{Gal2007B} used local shape descriptors on 3D
models to fill a 3D container with a ``collage'' in the style of
Arcimboldo.  Huang et al.~\cite{Huang2011} produced Arcimboldo-like
collages in 2D by layering objects cut out from images on top of a
segmented container.  These methods benefit from overlaps, which
join elements into a single large object.
Reinert et al.~\cite{Reinert2013} generated compositions by 
projecting objects from a high dimensional feature space down to 2D
while also inferring users' intentions when manually placing elements. However,
their goal was to create meaningful compositions without
an attempt to effectively fill a container.


As stated in the introduction, our work is most similar to Jigsaw
Image Mosaics (JIM)~\cite{Kim2002} and collages based on the Pyramid
of Arclength Descriptor PAD~\cite{Kwan2016}.  JIM packed
nearly-convex elements tightly by placing one element at a time and
backtracking as needed.  PAD developed a sophisticated shape
descriptor in order to find new elements that partially matched
existing element boundaries as a container was being filled.  Both
methods permitted some elements to overlap.  While they could
both correct gaps and overlaps using deformation, the deformation
was applied locally near edges in a post-processing step after
elements were frozen in place. 

\textbf{Text and letter packings:}
Xu and Kaplan~\cite{Xu2007} and Zou et al.~\cite{Zou2016}
constructed \textit{calligrams} by filling a container with a small
number of deformed letters composing one or two words.  Because the
order of the letterforms was defined by the text, their solutions
usually required significant distortion of the individual letters.
Their goal was to balance between filling the container and preserving
readability.  Maharik et al.~\cite{Maharik2011} explored Digital
Micrography, in which whole lines of text deform to fit along dense
streamlines in a flow field. Their results more closely resemble
textures than packings.

\textbf{Packings for fabrication:}
Related work in fabrication has sought to cover surfaces with
arrangements of deformed ornamental elements that satisfy manufacturing
constraints such as connectivity.  Chen et al.~\cite{Chen2016}
developed a method to synthesize filigree patterns out of simple
elements. 
In later work, Chen et al.~\cite{Chen2017}
generated modular surfaces by computing 
contact point networks of rigid elements.


Zehnder et al.~\cite{Zehnder2016} proposed an elegant method to
cover 3D surfaces with deformed ornamental elastic curves.
Our method has some similarities to theirs in that both start with 
scaled-down copies of elements and grow them, but the growth process is quite different.  
Unlike their approach, our elements exert
forces on each other throughout the growth process, allowing them a greater
opportunity to translate, rotate, and deform in search of more even negative
space.  Furthermore, the goal of their work (3D fabrication) is quite 
different from ours (2D ornamentation) and our results appear qualitatively
different.


\textbf{Non-rigid packing:}
Peng et al.~\cite{Peng2014} computed layouts by packing and deforming
simple polygons and polyominoes. Their method cannot handle more
complicated shapes, making it unsuitable for our style of packings.
FLOWPAK by Saputra et al.~\cite{Saputra2017} placed ornamental
elements to create a visual sense of flow. They used
skeletal strokes to place elements along streamlines defined from a
vector field.  However, their elements could not undergo more general
deformations, and their method did not explicitly control for the evenness
of the negative space.

\textbf{Physics-based NPR:} Pedersen and
Singh~\cite{Pedersen2006} grew curves to create organic
labyrinthine paths. Their algorithm is related to ours by the use
of repulsion forces to maintain even spacing and parallel segments.




