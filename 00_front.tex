% T I T L E   P A G E
% -------------------
% Last updated June 14, 2017, by Stephen Carr, IST-Client Services
% The title page is counted as page `i' but we need to suppress the
% page number. Also, we don't want any headers or footers.
\pagestyle{empty}
\pagenumbering{roman}

% The contents of the title page are specified in the "titlepage"
% environment.
\begin{titlepage}
        \begin{center}
        \vspace*{1.0cm}

        \Huge
        {\bf Deformation-Driven Element Packing }

        \vspace*{1.0cm}

        \normalsize
        by \\

        \vspace*{1.0cm}

        \Large
        Reza Adhitya Saputra \\

        \vspace*{3.0cm}

        \normalsize
        A thesis \\
        presented to the University of Waterloo \\ 
        in fulfillment of the \\
        thesis requirement for the degree of \\
        Doctor of Philosophy \\
        in \\
        Computer Science \\

        \vspace*{2.0cm}

        Waterloo, Ontario, Canada, 2020 \\

        \vspace*{1.0cm}

        \copyright\ Reza Adhitya Saputra 2020 \\


        
        Last edited: \today \;\; \currenttime

        \end{center}



\end{titlepage}

% The rest of the front pages should contain no headers and be numbered using Roman numerals starting with `ii'
\pagestyle{plain}
\setcounter{page}{2}

\cleardoublepage % Ends the current page and causes all figures and tables that have so far appeared in the input to be printed.
% In a two-sided printing style, it also makes the next page a right-hand (odd-numbered) page, producing a blank page if necessary.

 
% E X A M I N I N G   C O M M I T T E E (Required for Ph.D. theses only)
% Remove or comment out the lines below to remove this page
\begin{center}\textbf{Examining Committee Membership}\end{center}
  \noindent
The following served on the Examining Committee for this thesis. The decision of the Examining Committee is by majority vote.
  \bigskip
  
  \noindent
\begin{tabbing}
Internal-External Member: \=  \kill % using longest text to define tab length
External Examiner: \>  YYY YYY \\ 
\> Professor, Dept. of YYY, \\
\> University of YYY \\
\end{tabbing} 
  \bigskip
  
  \noindent
\begin{tabbing}
Internal-External Member: \=  \kill % using longest text to define tab length
Supervisor(s): \> Craig S. Kaplan \\
\> Associate Professor, School of Computer Science, \\ 
\> University of Waterloo \\
\end{tabbing}
  \bigskip
  
  \noindent
  \begin{tabbing}
Internal-External Member: \=  \kill % using longest text to define tab length
Internal Member: \> Christopher Batty \\
\> Associate Professor, School of Computer Science, \\ 
\> University of Waterloo \\
\end{tabbing}
  \bigskip
  
  \noindent
\begin{tabbing}
Internal-External Member: \=  \kill % using longest text to define tab length
Internal-External Member: \> YYY YYY \\
\> Professor, Dept. of YYY, \\
\> University of Waterloo \\
\end{tabbing}
  \bigskip
  
  \noindent
\begin{tabbing}
Internal-External Member: \=  \kill % using longest text to define tab length
Other Member(s): \> Daniel Vogel \\
\> Associate Professor, School of Computer Science, \\ 
\> University of Waterloo \\
\end{tabbing}

\cleardoublepage

% D E C L A R A T I O N   P A G E
% -------------------------------
  % The following is a sample Delaration Page as provided by the GSO
  % December 13th, 2006.  It is designed for an electronic thesis.
%\phantomsection
%\addcontentsline{toc}{chapter}{Author's Declaration}
\begin{center}\textbf{Author's Declaration}\end{center}
  \noindent

\newtext{
This thesis consists of materials which I authored or co-authored: 
see Statement of Contributions. }
This is a true copy of the thesis, 
including any required final revisions,
as accepted by my examiners

  \bigskip
  
  \noindent
I understand that my thesis may be made electronically available to the public.

\cleardoublepage


%\phantomsection
%\addcontentsline{toc}{chapter}{Statement of Contribution}
\begin{center}\textbf{Statement of Contribution}\end{center}
  \noindent

\newtext{
This thesis was built from previous publications authored by myself,
Dr.\ Craig S.\ \mbox{Kaplan}, and Dr.\ Paul Asente.
}

\newtext{
FLOWPAK in Chapter~\ref{chapter_flowpak} was initially conducted with 
Dr.\ Paul Asente, and Dr.\ Radom\'{\i}r M\v{e}ch at Adobe Research in San Jose, California.}
\nnewtext{FLOWPAK is also discussed in our publication ``FLOWPAK: Flow-Based Ornamental Element Packing''~\cite{Saputra2017} and in our patent ``Computerized Generation of Ornamental Designs by Placing Instances of Simple Shapes in Accordance With a Direction Guide''~\cite{Asente2020}.
}

\newtext{
RepulsionPak in Chapter~\ref{chapter_repulsionpak} and Quantitative Metrics in Chapter~\ref{chapter_qualitative_metrics} 
were taken from ``RepulsionPak: Deformation-Driven Element Packing with Repulsion Forces''~\cite{Saputra2018} and 
``Improved Deformation-Driven Element Packing with RepulsionPak''~\cite{Saputra2019}.
}

\newtext{
AnimationPak in Chapter~\ref{chapter_animationpak} was taken from ``AnimationPak: Packing Elements with Scripted Animations''~\cite{Saputra2020}.
}


\cleardoublepage


% A B S T R A C T
% ---------------
\phantomsection
\addcontentsline{toc}{chapter}{Abstract}
\begin{center}\textbf{Abstract}\end{center}

\nnewtext{A packing is an arrangement of geometric elements within a container region in the plane.
Elements are united to communicate the overall container shape,
but each is large enough to be appreciated individually.
Creating a packing is challenging since an artist should arrange compatible elements 
so that their boundaries interlock with each other.
This thesis presents three packing methods that create element compatibilities through shape deformation.}
\nnewtext{The first method, FLOWPAK, deforms} elements to flow along a vector field interpolated
from user-supplied strokes, giving a sense of visual flow to the final composition.
\nnewtext{The second method, RepulsionPak, utilizes} repulsion forces to pack
elements, each represented as a mass-spring system, allowing them to deform
to achieve a better fit with their neighbors and the container.
\nnewtext{The last method, AnimationPak, creates} animated packings
by arranging animated two-dimensional elements inside a static
container. We represent animated elements in a three-dimensional
spacetime domain, and view the animated packing problem as a
three-dimensional packing in that domain.
Finally, we propose statistical methods for measuring the evenness of
2D element distributions, which provide quantitative
means of evaluating and comparing packing algorithms.

\cleardoublepage

% A C K N O W L E D G E M E N T S
% -------------------------------

\begin{center}\textbf{Acknowledgements}\end{center}

I would like to thank all people who have helped me during my PhD study:


\begin{itemize}  
\item Craig S. Kaplan and Paul Asente, for being the best teachers.
\item Radom\'ir M\v{e}ch, for the internship and the help with FLOWPAK paper. 
\item Daichi Ito, for the discussion about design principles and using RepulsionPak to create a packing.
\item Dietrich Stoyan, for the discussion about spatial statistics.
\item Danny Kaufman, for the discussion about physical simulations.
\item Steve Mann, for giving me valuable lessons.
\item Bill Cowan, for his advice.
\item Christopher Batty and Dan Vogel, for being my PhD committee members.
\item Matt Thorne, Alex Pytel, and Tyler Nowicki, for all the lunches together. 
\item All my running friends in Kitchener-Waterloo and ultra running community in \mbox{Ontario}, 
who teach me on how to keep grinding.
\item Russ Jones, for permission to use the fish art in Figure~\ref{fig_dog_flow}d.
\item Yusuf Umar, for designing ornamental elements and the box bird in Figure~\ref{bird_square}.
\end{itemize}

\cleardoublepage

% D E D I C A T I O N
% -------------------

\begin{center}\textbf{Dedication}\end{center}

For my Mom, Dad, and sister.
\cleardoublepage

% T A B L E   O F   C O N T E N T S
% ---------------------------------
\renewcommand\contentsname{Table of Contents}
\tableofcontents
\cleardoublepage
\phantomsection    % allows hyperref to link to the correct page

% L I S T   O F   T A B L E S
% ---------------------------
\addcontentsline{toc}{chapter}{List of Tables}
\listoftables
\cleardoublepage
\phantomsection		% allows hyperref to link to the correct page

% L I S T   O F   F I G U R E S
% -----------------------------
\addcontentsline{toc}{chapter}{List of Figures}
\listoffigures
\cleardoublepage
\phantomsection		% allows hyperref to link to the correct page

% GLOSSARIES (Lists of definitions, abbreviations, symbols, etc. provided by the glossaries-extra package)
% -----------------------------
\printglossaries
\cleardoublepage
\phantomsection		% allows hyperref to link to the correct page

% Change page numbering back to Arabic numerals
\pagenumbering{arabic}

