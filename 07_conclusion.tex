
%%%%%%%%%%%%%%%%%%%%%%%%%%%%%%%%%%%%%%%%%%%%%%%%%%%%%%%%%%
\chapter{Conclusions and Future Work}
\label{chapter_conclusions_and_future_work}
%%%%%%%%%%%%%%%%%%%%%%%%%%%%%%%%%%%%%%%%%%%%%%%%%%%%%%%%%%

\section{Conclusions}

\newtext
{
We presented three deformation-driven packing methods.
FLOWPAK is a method to deform pack long thin elements
to follow a user-defined vector fields.
RepulsionPak is a method that uses repulsion forces to pack
elements that are represented as mass-spring systems.
AnimationPak is an extension of RepulsionPak that packs animated 2D elements,
each is an extruded 3D shape in a spacetime domain.
We show that deformation-driven methods can create element compatibilities.
}

\newtext
{
Given a small size element library, repeated elements
give a sense of uniformity but deformation creates a sense of variety.
As element compatibilities are increased,
the negative space is more even.
}
\newtext
{
We quantify the evenness of negative space using three statistical metrics:
spherical contact probabilities, histograms of distance transforms, and overlap functions.
}

\section{Future Work}

\newtext
{
We see many possibilities for further improvements to our methods and packing research.
}

\subsection{FLOWPAK}

%We see many possibilities for further improvements to our algorithm and
%future research on ornamental packing.  

\begin{itemize}

\item In current implementation, elements must completely fill streamlines.
It would be worthwhile to investigate whether multiple shorter elements
could be threaded along streamlines.

\item FLOWPAK results do not have significant high-curvature streamlines like u-turns, 
since they could unpleasantly fold the decorative elements. This could be solved with a folding avoidance algorithm~\cite{Asente2010}.

\item The iterative refinement process uses a greedy approach, in which we 
iterate over all placed elements in a fixed order from smallest to largest.
We would like to investigate global optimizations
that could be applied to improve the overall composition in an 
order-independent way.  A natural first choice would be an approach based
on simulated annealing, although performance could become a more serious
issue in that case.

\item We would like to explore the automatic creation and placement of 
the fixed elements, perhaps by discovering them as salient regions 
in source photographs, and extracting and vectorizing them.  This 
extraction must be carried out carefully, yielding enough fixed elements
to communicate a container clearly without disrupting the uniformity of
the design.

\end{itemize}

\subsection{RepulsionPak}

%We see many possibilities for further improvements to RepulsionPak
%and future research on element packings.

\begin{itemize}
\item 
\newtext{
	We deliberately chose a simple 
	simulation model based on springs and forward Euler integration,
	because our main goal was to demonstrate the validity of a 
	deformation-driven approach, and not to contribute a new
	physical simulation method.}
	Contemporary research has yielded many more sophisticated 
	physical simulation methods, such as Position Based Dynamics~\cite{Muller2007}, 
	Projective Dynamics~\cite{Bouaziz2014}, and the Finite Element Method.
	No one method is obviously best suited to this problem, and
	we intend to experiment with several to investigate if any offers
	a suitable trade-off between performance and quality.

\item We would like to explore the use of RepulsionPak in a fabrication context.
      For example, our boundary compatibilities might be used to create a connected object.
      Alternatively, it would be interesting to 3D print the 
      negative space, which is already connected,
	  leaving holes that surround the element shapes.

\item Our barycentric warping method can
	introduce undesirable artifacts in highly deformed elements, as
	in the swallow tails in the left result of Figure~\ref{three_packings}.
	We would like to explore other methods for warping an element's
	geometry based on the correspondences between the triangles of its original
	mesh and the deformed meshes in the final packing, based for example on the
	work of Jacobson et al.~\cite{Jacobson2011} and Liu et al.~\cite{Liu2014}.

\end{itemize}

\subsection{AnimationPak}

%We see an number of opportunities for improvements and extensions to
%AnimationPak:

\begin{itemize}
\item Because we use linear interpolation to synthesize an element's shape
	between slices, we require elements not to undergo changes in 
	topology.  More sophisticated representations of vector shapes,
	such as that of Dalstein et al.~\cite{Dalstein2015}, could support
	interpolations between slices with complex topological changes.
	We would also need to synthesize a watertight envelope around the
	animating element in order to compute overlap and repulsion forces.

\item We would like to improve the performance of the physical simulation.
	One option may be to increase the resolution of element meshes 
	progressively during simulation.  Early in the process, elements are
	small and distant from each other, so lower-resolution
	meshes may suffice for computing repulsion forces.

\item %As noted in Section~\ref{section_rendering} and Figure~\ref{fig_animationpak_blender},
	Our discrete simulation can miss element overlaps that occur between
	slices.  A more robust continuous collision detection (CCD) algorithm
	such as that of Brochu et al.~\cite{Brochu2012}
	could help us find all collisions between
	the envelopes of spacetime elements.

\item In RepulsionPak~\cite{Saputra2018}, an additional pass with
	small secondary elements had a significant positive effect on the
	distribution of negative space in the final packing.  It may be
	possible to identify stretches of unused spacetime that can be filled
	opportunistically with additional elements.  The challenge would be
	to locate tubes of empty space that run the full duration of the
	animation, always of sufficient diameter to accommodate an added
	element.

\item Like the spectral method~\cite{Dalal2006}, and unlike
	Animosaics~\cite{Smith2005}, AnimationPak can pack animated 
	elements into a static container.  We would like to extend
	our work to also handle animated containers. 
	\newtext
	{
	It would be interesting to investigate whether we could adapt to changes in
	container area by adding and removing elements, 
	we are certain that it would not create undesirable element scaling as demonstrated by Figure~\ref{fig_bib_entering_star}.
	However, this extension would certainly affect the initial element placement, which 
	would need to ensure that elements are placed fully inside the
	spacetime volume of the container.  
	}

\item AnimationPak implements forces and constraints geared towards 
	spacetime animation, but many of the same ideas could be adapted
	to develop a deformation-driven method for packing purely spatial
	3D objects into a 3D container.  We would like to evaluate the
	expressivity and visual quality of deformation-driven 3D packings 
	in comparison to other 3D packing techniques.

\item There are many examples of static two-dimensional packings
	created by artists, which canserve as inspiration for an algorithm like RepulsionPak.  
	We were only able to find a single example of animated packing from The Simpsons, 
	We think its unpopularity is because it is difficult and time-consuming to create by hand.
	We would like 	to engage with artists to understand the aesthetic value and limitations
	of AnimationPak.

\end{itemize}

\subsection{Packing Evaluation}

\begin{itemize}

\item We would like to conduct experiments that investigate the
  extent to which quantitative measurements of the evenness of negative
  space in a packing correlate with the human perception of a
  packing's quality.  In informal evaluations, some viewers found that
  the packing in Figure~\ref{balabolka_comparison}b, created with RepulsionPak,
  was packed more tightly than the artist's packing in
  Figure~\ref{balabolka_comparison}a, even though both have the same total
  amount of negative space.

%\item As with many research projects in non-photorealistic rendering, this
%work raises deep questions about the aesthetics of ornamental packings.
%What compositions are most appealing?  What is the most effective
%way to distribute negative space?  Some hand-drawn compositions insert additional small elements,
%like circles and squares, to break up large areas of negative space;
%an automated simulation of this process would be helpful.

\item We would like to develop additional metrics to evaluate
  how well an ornamental design fulfills other design principles.
  A measure of element deformation in a composition would permit 
  a comparison against future deformation-driven techniques.
  \newtext{In Chapter~\ref{chapter_flowpak}, we argue that visual flow and
  ``uniformity amidst variety'' are important to attractive packings.} 
  In another study, Wong et al.~\cite{Wong1998} describe basic design
  principles for decorative arts: repetition, balance, and conformation
  to geometric constraints.  The rigorous expression of aesthetic principles
  is a fascinating area for future research.

\item Our validation metrics are all based on Minkowski sums or
  differences with discs, corresponding to a form of shape offsetting
  where corners become round.  The result of this rounding is visible in
  our graphs, for example in the gradual flattening of the SCP for
  the squares in Figure~\ref{hsr_viz}e.  It would be worthwhile to repeat
  these measurements using mitered offsetting, and to evaluate whether
  rounded or mitered offsetting is a closer match to human perceptual
  judgment of evenness.

\item When comparing calibrated packings, SCPs communicate 
  differences in the evenness of negative space, but the differences 
  between SCPs can be subtle.  In addition to distance histograms, 
  we would like to investigate other visualizations of this information
  that might amplify these differences to make evaluation easier.

\item All three metrics are only for evaluating 2D packings.
  While they extend naturally to three purely spatial
  dimensions, it is not clear whether they can be 
  adapted to the spacetime context.  
  We would like to investigate
  spatial statistics for the quality of animated packings created by AnimationPak that correlate
  with human perceptual judgments.

\end{itemize}



\mynote{TODO:
\begin{packeditems}
\item Enlarge figures
\item Remove 1-2 rows on a page if figures are too big.
\item $t$ or $t_sim$ for RepulsionPak?
\item This related work part and other related work parts?
\item Landscape pages in AnimationPak? 
https://tex.stackexchange.com/questions/337/how-to-change-certain-pages-into-landscape-portrait-mode
\item streamline sl and slice s
\end{packeditems}
}
